\newpage
\section{A specimen of pure aluminum was deformed and then heated to $500^{\circ}$C, resulting in recrystallization. Assuming the dislocation density of the deformed aluminum is $10^{15}$ m$^{-2}$, calculate the radius of the recrystallization nucleus. Assume the nucleus is spherical.}

The critical nucleus size for recrystallization can be calculated with the following equation:
\begin{align}
    \label{eq:r_recrystallization}
    r_{crit}&=\dfrac{2\gamma}{P},
\end{align}
where $P$ is the driving force for recrystallization in Pa and $\gamma$ is the grain boundary energy in J/m$^2$ \citep[p.~283]{rollett2017recrystallization}.

The driving force $P$ for recrystallization is provided by the dislocation density $\rho$ which results in a stored energy, and it is given by:
\begin{align}
    \label{eq:drivin_force}
    P&=\alpha \rho Gb^2,
\end{align}
where $\alpha$ is a constant of value $0.5$, $G$ is the shear modulus in Pa, $b$ is the magnitude of the Burgers vector in m and $\rho$ is the dislocation density in m$^{-2}$ \citep[p.~249]{rollett2017recrystallization}.

The sheer modulus ($G$) value for Al is $25$ GPa as it is shown in table 6.1 of \citet[p.~157]{callister2010materials}.
%\shnote{me falta revisar lo del sheer modulous, creo que deberia de ser el valor para la temeratura a la que se esta trabajando}

The Burgers vector magnitude can be calculated from the Burgers vector, which for an FCC cell is given by:
\begin{align}
    \label{eq:fcc_vector}
    \mathbf{b}(\text{FCC})&=\dfrac{a}{2}\langle uvw\rangle=\dfrac{a}{2}\langle110\rangle
\end{align}
\citep[p.~204]{callister2010materials}.

The magnitude of the vector can be calculated using the following equation:
\begin{align}
    \label{eq:vector_magnitude}
    \left|\mathbf{b}\right|&=\dfrac{a}{2}\left(u^{2}+v^{2}+w^{2}\right)^{1/2}=\dfrac{a}{2}\left(1^{2}+1^{2}+0^{2}\right)^{1/2}=\dfrac{a}{\sqrt{2}},
\end{align}
where $a$ is the lattice constant, and for Aluminum its value is: $0.40494$ nm \citep{Xu2011}, which gives:
\begin{align}
    \label{eq:vector_magnitude01}
    \left|\mathbf{b}\right|&=\dfrac{0.40494}{\sqrt{2}}=0.28633582.
\end{align}

Using the values of $\alpha$, $\rho$, $G$, and $b$, the value for the driving force can be calculated, using equation \ref{eq:drivin_force}:
\begin{align}
    \label{eq:drivin_force_num}
    \begin{split}
        P&=\left(0.5\right)\left(10^{15}\right)\left(25\times10^9\right)\left(0.28633582\times10^{-9}\right)^2=10248525.225 \\
        P&\approx 10.2 \text{MPa}
    \end{split}
\end{align}

To calculate the critical nucleus size for recrystallization equation \ref{eq:r_recrystallization} is used, with the value of driving force ($P$) and the value of the grain boundary energy of $375$ mJ/m$^2$ \citep[p.~125]{rollett2017recrystallization}:
\begin{align}
    \label{eq:recrystallization_num}
    \begin{split}
        r_{crit}&=\dfrac{2*0.375}{10248525.225} = 7.318126\times10^{-8} \text{m} \\
        r_{crit}&\approx 73.1 \text{nm}
    \end{split}
\end{align}

\begin{mdframed}
    The nucleus size of recrystallization for aluminum is $73.1$ nm.
\end{mdframed}
%\shnote{esta parte me falta lo del valor de gamma que para esta ecuacion es la grain boundary energy, pero no se de donde sacarla (o no he encontrado donde estaria :c)}