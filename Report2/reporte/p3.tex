\newpage
\section{}

\textbf{A specimen of pure aluminium was deformed and then heated to $500^{\circ}$C, resulting in recrystallization. Assuming the dislocation density of the deformed aluminium is $1015$ m$^{-2}$, calculate the radius of the recrystallization nucleus. Assume the nucleus is spherical.}

The critical nucleus size for nucleation can be calculated with the following equation \citet{rollett2017recrystallization}:
\begin{align}
    \label{eq:r_recrystallization}
    r_{crit}&=\dfrac{2\gamma}{P},
\end{align}
where $P$ is the driving force for recrystalization and $\gamma$ is the grain boundary energy.

The driving force $P$ for recistalization is provided by the dislocation density $\rho$ which results in a stored energy, and it is given by:
\begin{align}
    \label{eq:drivin_force}
    P&=\alpha \rho Gb^2,
\end{align}
where $\alpha$ is a constant of value $0.5$, $G$ is the shear modulous, $b$ is the magnitude of the Burgers vector and $\rho$ is the dislocation density.

The sheer modulous valu

The Buergers vector magnitude can be calculated from the Burgers vector, which for an FCC cell is given by \citet{callister2010materials}:
\begin{align}
    \label{eq:fcc_vector}
    \mathbf{b}(\text{FCC})&=\dfrac{a}{2}\langle uvw\rangle=\dfrac{a}{2}\langle110\rangle.
\end{align}
The magnitude of the vector can be calculated using the following equation:
\begin{align}
    \label{eq:vector_magnitude}
    \abs{\mathbf{b}}&=\dfrac{a}{2}\left(u^{2}+v^{2}+w^{2}\right)^{1/2}=\dfrac{a}{2}\left(1^{2}+1^{2}+0^{2}\right)^{1/2}
\end{align}
\shnote{esta parte me falta lo del valor de gamma que para esta ecuacion es la grain boundary energy, pero no se de donde sacarla (o no he encontrado donde estaria :c)}