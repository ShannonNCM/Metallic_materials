\newpage
\section{}

\subsection{}
Pure water was quasi-statically cooled from room temperature to $-5^{\circ}$C under $1$ atm pressure. Calculate the critical nucleus radius under these conditions. Assume the nucleus is spherical, the latent heat is $6$ kJ/mol, the interfacial energy between ice and water is $30$ mJ/m$^2$, and the density of ice is $1$ g/cm$^3$.

The critical nucleus radius can be calculated using the following equation \citet{callister2010materials}: 
\begin{align}
  \label{eq:r01}
  r^*&=\dfrac{-2\gamma}{\Delta G_v},
\end{align}

where $\gamma$ is the surface free energy, and $\Delta G_v$ is the volume free energy change. Because $\Delta G_v$ is a function of temperature:
\begin{align}
  \label{eq:deltagv}
  \Delta G_v&= \dfrac{\Delta H_f\left(T_m-T\right)}{T_m}.
\end{align}

Substituting equation \ref{eq:deltagv} in \ref{eq:r01} gives: 
\begin{align}
  \label{eq:nucleus_radius}
  r^*&=\left(\dfrac{-2\gamma T_m}{\Delta H_f}\right)\left(\dfrac{1}{T_m - T}\right),
\end{align}
where $\gamma$ is the surface free energy, $\Delta H_f$ is the the latent heat of fusion, $T_m$ is the melting temperature and $T$ is the transformation temperature.

The critcal nucleus radius for water quasi-statically cooled from room temperature to $-5^{\circ}$ is:
\begin{align}
  \label{eq:radius}
  r^*&=\left(\dfrac{-2*0.03*273}{6000}\right)\left(\dfrac{1}{273-268}\right)=
\end{align}

\shnote{tengo que arreglar lo del calculo, porque el calor latente deberia de estar en kj/m3 y yo lo tengo en J/m2 asi que algo tengo que hacerle con la densidad pero no se que? :c}

\subsection{}
Explain how the critical nucleus radius changes if the pure water is
further cooled, including the reason.

To visualize the changes of the radius with further cooling water, the critical radius was calculated for different temperatures, from $-5$ to $-200^{\circ}$ C. The results are shown in figure: \shnote{Estoy haciendo la grafica de estos calculos}